\newcommand{\sheet}{5}
\documentclass{article}

\usepackage[english, german]{babel}
\usepackage{amsthm,amssymb,amsmath,mathrsfs,mathtools}
\usepackage[shortlabels]{enumitem}
\usepackage{algpseudocode}
\usepackage{hyperref}
\usepackage{tikz}
\usepackage{tikz-cd}

% \usepackage[tmargin=1.25in,bmargin=1.25in,lmargin=1.2in,rmargin=1.2in]{geometry}


\newcommand{\C}{\mathbb{C}}
\newcommand{\R}{\mathbb{R}}
\newcommand{\N}{\mathbb{N}}
\newcommand{\Q}{\mathbb{Q}}
\newcommand{\Z}{\mathbb{Z}}

\DeclareMathOperator{\id}{id}
\DeclareMathOperator{\im}{im}
\DeclareMathOperator{\GL}{GL}
\DeclareMathOperator{\sgn}{sgn}
\DeclareMathOperator{\Tor}{Tor}
\DeclareMathOperator{\Sym}{Sym}
\DeclareMathOperator{\coker}{coker}
\DeclareMathOperator{\Quot}{Quot}
\DeclareMathOperator{\supp}{supp}
\DeclareMathOperator{\Hom}{Hom}
\DeclareMathOperator{\Spec}{Spec}
\DeclareMathOperator{\MinSpec}{MinSpec}
\DeclareMathOperator{\argmin}{argmin}


\newenvironment{exercise}[1] {
  \vspace{0.5cm}
  \noindent \textbf{Exercise~{\sheet}.{#1}.}
} {
  \vspace{0.5cm}
}
\newenvironment{claim} { \par\noindent \textbf{Claim.} } { }
\newenvironment{proof_claim} { \par\noindent \textbf{Proof of claim.} } { \qed{} (of claim) }

\title{Combinatorial optimization\\Exercise sheet \sheet}
\author{Solutions by: Anjana E Jeevanand and David Čadež}

\date{\today}


\begin{document}

\maketitle

\begin{exercise}{3}
    Let $G$ a graph and $T \subseteq V(G)$ with $|T|$ even.
    \begin{enumerate}[i)]
        \item{
                Suppose the set $F \subseteq E(G)$ intersects every $T$-join.
                Take some $t \in T$. Now look at all $v \in V(G)$ for which
                there exists a $t$-$v$-path that does not intersect $F$:
                \begin{equation*}
                    X = \{ v \in V(G) \mid \text{exists a $t$-$v$-path that does
                    not intersect $F$} \} \cup \{t\}
                \end{equation*}

                Use induction.

                Suppose $T = \{t, s\}$. Then $T$-joins are exactly
                $t$-$s$-paths. We are assuming every such path intersects $F$.
                In this case the set $X$ then only contains one element, and it
                clearly defines a $T$-cut $C = \delta(X)$, because every edge
                $\{u, v\} \in C$ with $u \in X$ and $v \notin X$ it must hold
                $\{u, v\} \in F$, otherwise we would get $v \in X$.

                And the induction step:

                \begin{itemize}
                    \item{If $|X \cap T|$ odd, then we found a $T$-cut that is
                        contained in $F$, namely the cut $C = \delta(X)$. Lets
                        quickly argument why this is true: every edge $\{u, v\}
                        \in C$ (with $u \in X$ and $v \notin X$) has to lie in
                        $F$, otherwise we could extend the path from $t$ to $u$
                        with the edge $\{u, v\}$ and get $v \in X$.}
                    \item{If $|X \cap T|$ even, then for every $t_i \in X \cap
                        T$, $t_i \not= t$, define the path $p_i$ to be the path
                        from $t$ to $t_i$ that does not intersect $F$. Take
                        a symmetric difference of all these paths
                        $\otimes p_i$, which is,
                        using a proposition from the lectures, a $T \cap
                        X$-join. Let $T' = T \setminus X$. If there existed a
                        $T'$-join $J$ would not intersect $F$, then $\otimes p_i
                        \otimes J$ would be a $T$-join that does not intersect
                        $F$. So every $T'$-join intersects $F$. Using inductions
                        hypothesis we get that there exists a $T'$-cut $C =
                        \delta(X')$, which is contained in $F$. Clearly $t \in
                        X'$ if and only if $T \cap X \subseteq X'$, so $|X'
                        \cap T|$ is odd, which proves it is a $T$-cut.
                        }
                \end{itemize}
                In both cases we found a $T$-cut that is contained in $F$.

                The other direction is obvious using the proposition from the
                lectures, which says that every $T$-cut and $T$-join intersect.}
        \item{First remember the proposition from the lectures, saying that a
            subgraph $G' \subseteq G$ contains a $T$-join if and only if every
            connected component of $G'$ contains even number of elements in $T$. 

            Define $V' = \cup_{e \in F} e$ and a graph $G' = (V', F)$. Suppose
            there exists a connected component $C' \subseteq G'$ for which
            $|V(C') \cap T|$ is odd. Then simply taking $C := \delta(V(C'))$ is
            a $T$-cut that does not intersect $F$ (because $C'$ is by definition
            a connected component in $G'$).

            The other direction is again obvious using the proposition from the
            lectures, which says that every $T$-cut and $T$-join intersect.}
    \end{enumerate}
\end{exercise}

\end{document}
