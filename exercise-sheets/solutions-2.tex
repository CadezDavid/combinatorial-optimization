\newcommand{\sheet}{2}
\documentclass{article}

\usepackage[english, german]{babel}
\usepackage{amsthm,amssymb,amsmath,mathrsfs,mathtools}
\usepackage[shortlabels]{enumitem}
\usepackage{algpseudocode}
\usepackage{hyperref}
\usepackage{tikz}
\usepackage{tikz-cd}

% \usepackage[tmargin=1.25in,bmargin=1.25in,lmargin=1.2in,rmargin=1.2in]{geometry}


\newcommand{\C}{\mathbb{C}}
\newcommand{\R}{\mathbb{R}}
\newcommand{\N}{\mathbb{N}}
\newcommand{\Q}{\mathbb{Q}}
\newcommand{\Z}{\mathbb{Z}}

\DeclareMathOperator{\id}{id}
\DeclareMathOperator{\im}{im}
\DeclareMathOperator{\GL}{GL}
\DeclareMathOperator{\sgn}{sgn}
\DeclareMathOperator{\Tor}{Tor}
\DeclareMathOperator{\Sym}{Sym}
\DeclareMathOperator{\coker}{coker}
\DeclareMathOperator{\Quot}{Quot}
\DeclareMathOperator{\supp}{supp}
\DeclareMathOperator{\Hom}{Hom}
\DeclareMathOperator{\Spec}{Spec}
\DeclareMathOperator{\MinSpec}{MinSpec}
\DeclareMathOperator{\argmin}{argmin}


\newenvironment{exercise}[1] {
  \vspace{0.5cm}
  \noindent \textbf{Exercise~{\sheet}.{#1}.}
} {
  \vspace{0.5cm}
}
\newenvironment{claim} { \par\noindent \textbf{Claim.} } { }
\newenvironment{proof_claim} { \par\noindent \textbf{Proof of claim.} } { \qed{} (of claim) }

\title{Combinatorial optimization\\Exercise sheet \sheet}
\author{Solutions by: Anjana E Jeevanand and David Čadež}

\date{\today}


\begin{document}

\maketitle

\begin{exercise}{1}
    \begin{claim}
        The number of ears in an odd decomposition of a graph is uniquely
        determined by the following value:
        \[
            \text{\# of ears} = 1 + \frac{1}{2} \sum_{v \in V(G)} (\deg(v) - 2).
        \]
    \end{claim}
    \begin{proof_claim}
        With induction on the number of ears. Statement holds if $G$
    is an odd circuit. Let $G = (\{r\}, \emptyset) + P_1 + \ldots + P_k$ be an
    odd ear decomposition. For subgraph $G' = (\{r\}, \emptyset) + P_1 + \ldots
    + P_{k-1}$ then holds 
    \[
        k - 1 = 1 + \frac{1}{2} \sum_{v \in V(G')} (\deg(v) - 2).
    \]
    By adding $P_k$ to $G'$ we add only vertices of degree $2$ (w.r.t.\ graph
    $G$), but we do increase the degree of $2$ vertices (where path $P_k$
    connects to $G'$) by $1$. Therefore
    \begin{equation*}
        \frac{1}{2} \sum_{v \in V(G)} (\deg(v) - 2) = 1 + \frac{1}{2} \sum_{v
        \in V(G')} (\deg(v) - 2) = k
    \end{equation*}
    which proves the claim.
    \end{proof_claim}

    Claim proves that the number of ears is uniquely determined by the graph.
\end{exercise}

\begin{exercise}{4}
    We have a graph $G$, with $n := |V(G)|$ even, and for any set $X \subseteq
    V(G)$ with $|X| \leq \frac{3}{4} n$ we have
    \begin{equation}\label{condition}
        \left| \bigcup_{x \in X} \Gamma(x) \right| \geq \frac{4}{3} |X|.
    \end{equation}
    We have to prove that $G$ has a perfect matching.

    Suppose it does not have a perfect matching. Then by Tutte's theorem there
    exists a set $S \subseteq V(G)$ that violates Tutte condition, i.e. $q_G(S)
    > |S|$.

    First make the following simplification: let the set $S$ be maximal, in the
    sense that there exists no other $S' \supseteq S$ with $q_G(S') > |S'|$. A
    direct consequence of that is that $G \setminus S$ does not have even
    connected components. If it did have an even connected component $C$, we
    could take any $v \in C$ and define $S' := S \cup \{v\}$. Then $C \setminus
    \{v\}$ would contain at least one odd connected component, simply because $C
    \setminus \{v\}$ has odd number of vertices. This will be useful when
    considering the case $|S| < \frac{n}{4}$.

    \begin{itemize}
        \item{}
        First consider the case when $|S| \geq \frac{n}{4}$. Define the set
        \[
            I = \{ v \in G \setminus S \mid v\ \text{isolated in}\ G \setminus S \}.
        \]
        Using condition~\ref{condition} on the set $G \setminus S$ we get
        \begin{equation}\label{equation_1}
            \frac{4}{3} |G \setminus S| \leq \left|\bigcup_{x \in G \setminus S} \Gamma(x)
            \right|.
        \end{equation}

        Observe that
        \begin{equation*}
            \bigcup_{x \in G \setminus S} \Gamma(x) \subseteq G \setminus I,
        \end{equation*}
        because elements of $I$ are by definition isolated in $G \setminus S$ and
        therefore have no neighbors in $G \setminus S$. Therefore
        \begin{equation}\label{equation_2}
            \left|\bigcup_{x \in G \setminus S} \Gamma(x) \right| \leq | G \setminus
            I | = n - |I|.
        \end{equation}

        We input equation~\ref{equation_2} in equation~\ref{equation_1} to get
        \begin{equation}
            \frac{4}{3} (n - |S|) \leq n - |I|
        \end{equation}
        and thus
        \begin{equation}\label{equation_3}
            |I| \leq \frac{4}{3} |S| - \frac{1}{3} n.
        \end{equation}

        Now we count all vertices in $G$ in the following way
        \begin{equation}\label{count_vertices}
            |I| + 3 \left(q_G(S) - |I|\right) + |S| \leq n.
        \end{equation}
            Here we summed isolated vertices ($|I|$), vertices contained in odd components of $G
        \setminus S$ (using that their size is at least $3$ and there is at least
        $q_G(S) - |I|$ of them) and elements in the set $S$.

        We use equation~\ref{equation_3} and inequality $q_G(S) > |S|$ to
        calculate
        \begin{align*}
            |I| + 3 \left(q_G(S) - |I|\right) + |S| &\leq n \\
            -2 |I| + 3 q_G(S) + |S| &\leq n \\
            -2 \left( \frac{4}{3} |S| - \frac{1}{3} n \right) + 3 q_G(S) + |S|
            &\leq n \\
            -2 \left( \frac{4}{3} |S| - \frac{1}{3} n \right) + 4 |S| &< n \\
            - \frac{8}{3} |S| + \frac{2}{3} n + 4 |S| &< n \\
            |S| &< \frac{n}{4} \\
        \end{align*}
            which is contradiction with assumption $|S| \geq \frac{n}{4}$.

        \item{}
        Suppose now $|S| < \frac{n}{4}$. First prove the following claim.
        \begin{claim}\label{claim_1}
            Let $G$ be a graph with $n := V(G)$ even. Assume it satisfies
            condition~\ref{condition}. Then for every set $T \subseteq V(G)$
            with $|T| \leq \frac{n}{4}$ the subgraph $G \setminus T$ has no
            isolated vertices.
        \end{claim}
        \begin{proof_claim}
            We suppose there exists a set $T$ with $|T| \leq \frac{n}{4}$ such
            that $G \setminus T$ has an isolated vertex. Fix any isolated vertex
            $v \in G \setminus T$.

            Consider cases when $n$ divisible by $4$ and when it is not separately.
            \begin{itemize}
                \item{} Let $n = 4m$ for some $m \in \N$. We will pick a subset $X
                    \subseteq G \setminus T$ of exactly the size $3m$
                    (it exists because $T \leq \frac{n}{4}$). It fulfils the
                    condition $|X| \leq \frac{3}{4} n = 3m$, and therefore by
                    assumption
                    \begin{equation*}
                        \left| \bigcup_{x \in X} \Gamma(x) \right| \geq \frac{4}{3}
                        |X|.
                    \end{equation*}
                    But left side is at most $n - 1$, because $v$ is not a neighbor
                    of any element in $X$, and the right one is exactly $4m$. That
                    would mean $n - 1 \geq 4m = n$, so we arrive at a contradiction.
                \item{} Let $n = 4m + 2$ for some $m \in \N$. We pick a subset $X
                    \subseteq G \setminus T$ of exactly the size $3m +
                    1$ (it exists because $T \leq \frac{n}{4}$). Since $3m + 1 \leq
                    \frac{3}{4}n$, the set $X$ fulfils condition $|X| \leq
                    \frac{3}{4} n$, and therefore we must have
                    \begin{equation*}
                        \left| \bigcup_{x \in X} \Gamma(x) \right| \geq \frac{4}{3}
                        |X|.
                    \end{equation*}
                    But left side is at most $n - 1$, because $v$ is not a neighbor
                    of any element in $X$, and the right one is exactly
                    $\frac{4}{3}(3m + 1) = 4m + \frac{4}{3}$. That would mean
                    \begin{equation*}
                        n - 1 \geq 4m + \frac{4}{3} = n - 1 + \frac{1}{3},
                    \end{equation*}
                    so we arrive at a contradiction.
            \end{itemize}
            This proves the claim.
        \end{proof_claim}

        The claim therefore show that the subgraph $G \setminus S$ cannot have
        isolated vertices.

        We prove another claim.
        \begin{claim}
            Assume current environment variables, mainly $|S| < \frac{n}{4}$.
            Then for every connected component $C$ of $G \setminus S$ we have
            \begin{equation*}
                \frac{n}{4} - |S| < |C| - 1.
            \end{equation*}
        \end{claim}
        \begin{proof_claim}
            Suppose the statement wouldn't hold. Let $C$ be a connected component in
            $G \setminus S$ with $\frac{n}{4} - |S| \geq |C| - 1$. Pick an element $v \in
            C$. Then the set $S' := S \cup (C \setminus \{v\})$ violates first
            claim, because $v$ by assumption does not have any neighbors in the set
            $G \setminus S'$ and is thus an isolated vertex in $G \setminus S'$, but
            $|S'| \leq \frac{n}{4}$ also holds. This proves the claim.
        \end{proof_claim}

        Note that $q_G(S) \geq |S| + 2$, because $q_G(S)$ and $|S|$ have same
        parity (i.e. $q_G(S) - |S| = 0 \mod 2$), which we
        argued during the lectures already when we proved Tutte's theorem. It is a
        direct consequence of $V(G)$ being even and all components being odd.

        We consider cases when $n$ is divisible by $4$ and when it is not
        separately.

        Consider $n = 4m$ for some $m \in \N$.
        We make a simple estimate for amount of vertices in $G \setminus S$:
        \begin{equation*}
            n - |S| \geq 3 (|S| + 2)
        \end{equation*}
        which simplifies to $n \geq 4 |S| + 6$ and further to $m \geq |S| +
        \frac{3}{2}$. Since all involved variables are natural numbers, we must have
        $m \geq |S| + 2$. We can now show that every connected component in $G \setminus
        S$ has to be at least of size $5$. Using last claim we have that the value $\frac{n}{4} -
        |S| + 1 \geq |S| + 2 - |S| + 1 = 3$ must be strictly less than the size of
        any component. So all components must be of cardinality at least $5$.

        Therefore we have $n - |S| \geq 5 (|S| + 2)$ (simply by giving rough lower
        bound for amount of vertices in $G \setminus S$).

        Using all the things we calculated by now we can count vertices in $G$
        once again to obtain
        \begin{equation}\label{equation_6}
            |S| + (|S| + 2) \left( \frac{n}{4} - |S| + 2 \right) \leq n
        \end{equation}
        where we sum $|S|$ and product of amount of connected components with
        lower bound for their size. Manipulating this inequality gives:
        \begin{align*}
            |S| + (|S| + 2) \left( \frac{n}{4} - |S| + 2 \right) &\leq n \\
            |S| + |S| \frac{n}{4} - |S|^2 + 4 &\leq \frac{n}{2} \\
            (|S| - 2) \frac{n}{4} &\leq |S|^2 - |S| - 4 \\
            \frac{n}{4} &\leq \frac{|S|^2 - |S| - 4}{|S| - 2} \\
            \frac{n}{4} &\leq |S| + 2 - \frac{|S|}{|S| - 2}.
        \end{align*}


        Lets treat edge cases, since we divided by $|S| - 2$ in the calculation,
        which could in general be non-positive.
        \begin{itemize}
            \item{If $|S| = 0$, then $G$ would not be connected, which would
                clearly violate condition~\ref{condition} by picking $X$ to be
                the smallest component.}
            \item{If $|S| = 1$, then using condition~\ref{condition} on any connected
                component $C \subseteq G \setminus S$ yields $|C| \leq 3$. At
                the same time there are no isolated vertices in $G \setminus S$,
                so all connected components in $G \setminus S$ must be of size
                $3$. And there are at least $3$ connected components in $G
                \setminus S$, since $S$ violates Tutte condition. Pick $X$ to be
                a union of two components in $G \setminus S$, so $|X| = 6$. Then
                \begin{equation*}
                    \frac{4}{3} |X| \leq | \bigcup_{x \in X} \Gamma(x) | \leq |X
                    \cup S| = 7
                \end{equation*}
                gives contradiction in the case $|S| = 1$.
                }
            \item{If $|S| = 2$, then equation~\ref{equation_6} yields $2 \leq
                0$.}
        \end{itemize}

        Then we just plug in $n \geq 6 |S| + 10$ which we calculated earlier to get
        \begin{align*}
            \frac{6|S| + 10}{4} &\leq |S| + 2 - \frac{|S|}{|S| - 2} \\
            \frac{1}{2}|S| + \frac{1}{2} &\leq - \frac{|S|}{|S| - 2}
        \end{align*}
        another contradiction, this time with existance of such $S$ in case when
        $n$ is divisible by $4$.

        Suppose now $n$ is not divisible by $4$. It is even, so it must of the
        form $n = 4m + 2$ for some $m \in \N$.
        We again make an estimate for amount of vertices in $G \setminus S$.
        \begin{align*}
            n - |S| &\geq 3(|S| + 2) \\
            n &\geq 4|S| + 6
        \end{align*}
        which gives estimate $m \geq |S| + 1$. That is ``worse estimate'' than
        we got in case when $n$ was divisible by $4$. But if there would exist a
        connected component in $G \setminus S$ with at least $5$ elements, then
        we can get estimate $|S| + 2 \leq m$ as in the case when $n$ was
        divisible by $4$. Concretely we get
        \begin{align*}
            |S| + 3 (|S| + 1) + 5 &\leq 4m + 2 \\
            |S| + \frac{3}{2} &\leq m. \\
            |S| + 2 &\leq m
        \end{align*}
        If there are more than $|S| + 2$ connected components in $G \setminus
        S$, we also get estimate $|S| + 2 \leq m$ just like in the case when $n$
        was divisible by $4$, concretely
        \begin{align*}
            |S| + 3 (|S| + 3) &\leq 4m + 2 \\
            |S| + \frac{7}{4} &\leq m \\
            |S| + 2 &\leq m.
        \end{align*}
        So in these two cases (when $G \setminus S$ has either more connected
        components than $|S| + 2$ or it has one of size more than $5$), we do
        exact same argument as when $n$ was divisible by $4$. That argument did
        not use that $n$ is divisible by $4$ from that point on. Therefore
        arrive at the contradiction with such $S$ existing and $n$ being of the
        form $4m + 2$ and $G \setminus S$ either having a component at least of
        size $5$ or having strictly more than $|S| + 2$ connected components.

        Now we focus on the last case, if $G \setminus S$ only contains components
        of size $3$ and exactly $|S| + 2$ of them. Then $n = |S| + 3(|S| + 2)$, so
        $m = |S| + 1$. Pick any connected component of the subgraph $G \setminus S$
        and denote it with $C$. We use the condition~\ref{condition} on the set
        $X := G \setminus (S \cup C)$. Note that $C$ does not have any neighbors in
        other connected components in $G \setminus S$, so
        \begin{equation*}
            \left| \bigcup_{x \in X} \Gamma(x) \right| \leq n - 3
        \end{equation*}
        We also know $|X| = 3 (|S| + 1)$. Using condition we get
        \begin{align*}
            \frac{4}{3} |X| &\leq n - 3 \\
            4 (|S| + 1) &\leq n - 3 \\
            4m &\leq 4m + 2 - 3 \\
            4m &\leq 4m - 1
        \end{align*}
        which contradicts the fact that $S$ violating Tutte condition with $|S|
        < \frac{n}{4}$ exists in $G$.

        We covered all cases now, so there is no set $S$ that would violate
        Tutte condition, which means $G$ has a perfect matching by Tutte's
        theorem.
    \end{itemize}

    % But this is
    % absurd, it is not possible components would be that big. Lets count vertices
    % in $G \setminus S$ again
    % \begin{equation*}
    %     (|S| + 2) \left(\frac{1}{2} |S| + \frac{7}{2}\right) \leq n - |S|
    % \end{equation*}

    % We now observe the sizes of components in $G \setminus S$. We will show that
    % they have to be $\geq \frac{n}{4} - |S|$, otherwise we could add all but one
    % vertex of that component to $S$ and obtain a set smaller than $\frac{n}{4}$
    % whose complement would have an isolated point. Using this size restriction
    % on components we will arrive to contradiction of $|S| < \frac{n}{4}$.

    % Suppose there is a connected component $C \subseteq G \setminus S$ with
    % \begin{equation}\label{equation_4}
    %     |C| \leq \frac{n}{4} - |S|.
    % \end{equation}
    % Pick any $v \in C$ and define $S' := S \cup (C \setminus \{v\})$. Then $S'$
    % contradicts the statement in the claim above, because by construction $|S'|
    % < \frac{n}{4}$ and $G \setminus S'$ has an isolated point $v$.

    % So for all connected components $C$ we have $\frac{n}{4} - |S| < |C|$.
    % Let $C_0$ be the smallest connected component in $G \setminus S$. Then we
    % have inequality
    % \begin{equation}\label{equation_5}
    %     q_G(S)|C_0| \leq n - |S|
    % \end{equation}
    % because the left side counts the size of smallest component multiplied by
    % amonut of components (giving a lower estimate for amount of vertices in $G
    % \setminus S'$), meanwhile on the left we have the actual amount of vertices
    % in $G \setminus S'$.

    % Then we put equation~\ref{equation_4} into equation~\ref{equation_5} to
    % obtain
    % \begin{equation}\label{equation_6}
    %     q_G(S) \left( \frac{n}{4} - |S| \right) \leq n - |S|
    % \end{equation}
    % and using $q_G(S) > |S|$ we finally get
    % \begin{equation}\label{equation_6}
    %     |S| \left( \frac{n}{4} - |S| \right) < n - |S|
    % \end{equation}
    % which we can rearrange to get
    % \begin{equation}\label{equation_6}
    %     \frac{|S| - 4}{4} n < |S|^2 - |S|
    % \end{equation}
    % and further
    % \begin{equation}\label{equation_6}
    %     \frac{|S| - 4}{4} n < |S|^2 - |S|
    % \end{equation}
\end{exercise}

\end{document}
