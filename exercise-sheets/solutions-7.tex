\newcommand{\sheet}{7}
\documentclass{article}

\usepackage[english, german]{babel}
\usepackage{amsthm,amssymb,amsmath,mathrsfs,mathtools}
\usepackage[shortlabels]{enumitem}
\usepackage{algpseudocode}
\usepackage{hyperref}
\usepackage{tikz}
\usepackage{tikz-cd}

% \usepackage[tmargin=1.25in,bmargin=1.25in,lmargin=1.2in,rmargin=1.2in]{geometry}


\newcommand{\C}{\mathbb{C}}
\newcommand{\R}{\mathbb{R}}
\newcommand{\N}{\mathbb{N}}
\newcommand{\Q}{\mathbb{Q}}
\newcommand{\Z}{\mathbb{Z}}

\DeclareMathOperator{\id}{id}
\DeclareMathOperator{\im}{im}
\DeclareMathOperator{\GL}{GL}
\DeclareMathOperator{\sgn}{sgn}
\DeclareMathOperator{\Tor}{Tor}
\DeclareMathOperator{\Sym}{Sym}
\DeclareMathOperator{\coker}{coker}
\DeclareMathOperator{\Quot}{Quot}
\DeclareMathOperator{\supp}{supp}
\DeclareMathOperator{\Hom}{Hom}
\DeclareMathOperator{\Spec}{Spec}
\DeclareMathOperator{\MinSpec}{MinSpec}
\DeclareMathOperator{\argmin}{argmin}


\newenvironment{exercise}[1] {
  \vspace{0.5cm}
  \noindent \textbf{Exercise~{\sheet}.{#1}.}
} {
  \vspace{0.5cm}
}
\newenvironment{claim} { \par\noindent \textbf{Claim.} } { }
\newenvironment{proof_claim} { \par\noindent \textbf{Proof of claim.} } { \qed{} (of claim) }

\title{Combinatorial optimization\\Exercise sheet \sheet}
\author{Solutions by: Anjana E Jeevanand and David Čadež}

\date{\today}


\begin{document}

\maketitle

\begin{exercise}{1}
    Hint already gives us the graph, we just have to prove it satisfies the
    requirements. So let $(K_n, c)$ be a graph with weights $c(\{i, j\}) =
    \lambda_{i, j}$ (we use $\lambda_{i, j} = \lambda_{j, i}$ $\forall i, j \in
    K_n$ for this to be well-defined) and $T$ be a maximum weight spanning tree
    in $(K_n, c)$. Lets show local edge-connectivities in $T$ are exactly
    $\lambda_{i, j}$.

    Since $T$ is a tree, local edge-connectivity for any pair of vertices is the
    minimum of weights of edges on the path between them.

    Take $i, j \in T$.

    Condition $\lambda_{i, k} \geq \min\{\lambda_{i, j} \lambda_{j, k}\}$
    clearly implies $\lambda_{i, k} \geq \min_{e \in P_{i, k}} \lambda_e$, where
    $P_{i, k}$ is the edge set of the path between $i$ and $k$. This already
    proves that local edge-connectivity for a pair $i, j$ is smaller or equal to
    $\lambda_{i, j}$.

    Now suppose the inequality would be strict. Let $\{k, l\} \in P_{i, j}$ be
    an edge on the path between $i$ and $j$ with $\lambda_{k, l} < \lambda_{i,
    j}$. Then we could simply replace $\{k, l\} \in T$ with $\{i, j\}$ and
    obtain a tree with strictly bigger weight, which contradicts our assumption
    that $T$ is maximum weight.

    Therefore local edge-connectivity is exactly $\lambda_{i, j}$ for every $i,
    j \in T$.
\end{exercise}

\end{document}
