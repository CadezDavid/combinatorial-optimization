\newcommand{\sheet}{7}
\documentclass{article}

\usepackage[english, german]{babel}
\usepackage{amsthm,amssymb,amsmath,mathrsfs,mathtools}
\usepackage[shortlabels]{enumitem}
\usepackage{algpseudocode}
\usepackage{hyperref}
\usepackage{tikz}
\usepackage{tikz-cd}

% \usepackage[tmargin=1.25in,bmargin=1.25in,lmargin=1.2in,rmargin=1.2in]{geometry}


\newcommand{\C}{\mathbb{C}}
\newcommand{\R}{\mathbb{R}}
\newcommand{\N}{\mathbb{N}}
\newcommand{\Q}{\mathbb{Q}}
\newcommand{\Z}{\mathbb{Z}}

\DeclareMathOperator{\id}{id}
\DeclareMathOperator{\im}{im}
\DeclareMathOperator{\GL}{GL}
\DeclareMathOperator{\sgn}{sgn}
\DeclareMathOperator{\Tor}{Tor}
\DeclareMathOperator{\Sym}{Sym}
\DeclareMathOperator{\coker}{coker}
\DeclareMathOperator{\Quot}{Quot}
\DeclareMathOperator{\supp}{supp}
\DeclareMathOperator{\Hom}{Hom}
\DeclareMathOperator{\Spec}{Spec}
\DeclareMathOperator{\MinSpec}{MinSpec}
\DeclareMathOperator{\argmin}{argmin}


\newenvironment{exercise}[1] {
  \vspace{0.5cm}
  \noindent \textbf{Exercise~{\sheet}.{#1}.}
} {
  \vspace{0.5cm}
}
\newenvironment{claim} { \par\noindent \textbf{Claim.} } { }
\newenvironment{proof_claim} { \par\noindent \textbf{Proof of claim.} } { \qed{} (of claim) }

\title{Combinatorial optimization\\Exercise sheet \sheet}
\author{Solutions by: Anjana E Jeevanand and David Čadež}

\date{\today}


\begin{document}

\maketitle

\begin{exercise}{1}
    Hint already gives us the graph, we just have to prove it satisfies the
    requirements. So let $(K_n, c)$ be a graph with weights $c(\{i, j\}) =
    \lambda_{i, j}$ (we use $\lambda_{i, j} = \lambda_{j, i}$ $\forall i, j \in
    K_n$ for this to be well-defined) and $T$ be a maximum weight spanning tree
    in $(K_n, c)$. Lets show local edge-connectivities in $T$ are exactly
    $\lambda_{i, j}$.

    Since $T$ is a tree, local edge-connectivity for any pair of vertices is the
    minimum of weights of edges on the path between them.

    Take $i, j \in T$.

    Condition $\lambda_{i, k} \geq \min\{\lambda_{i, j} \lambda_{j, k}\}$
    clearly implies $\lambda_{i, k} \geq \min_{e \in P_{i, k}} \lambda_e$, where
    $P_{i, k}$ is the edge set of the path between $i$ and $k$. This already
    proves that local edge-connectivity for a pair $i, j$ is smaller or equal to
    $\lambda_{i, j}$.

    Now suppose the inequality would be strict. Let $\{k, l\} \in P_{i, j}$ be
    an edge on the path between $i$ and $j$ with $\lambda_{k, l} < \lambda_{i,
    j}$. Then we could simply replace $\{k, l\} \in T$ with $\{i, j\}$ and
    obtain a tree with strictly bigger weight, which contradicts our assumption
    that $T$ is maximum weight.

    Therefore local edge-connectivity is exactly $\lambda_{i, j}$ for every $i,
    j \in T$.
\end{exercise}

\begin{exercise}{3}
    % Lets start with necessity.
    % Let $f$ be a perfect $b$-matching in $(G, u)$.
    % Suppose we have disjoint $X, Y \subseteq V(G)$ for which number of connected
    % components $C$, with $b(C) + u(E_G(V(C), Y))$ odd, exceeds
    % \begin{equation*}
    %     \sum_{v \in X} b(v) + \sum_{y \in Y} \left( \sum_{e \in \delta(y)} u(e)
    %     - b(y) \right) - \sum_{e \in E_G(X, Y)} u(e).
    % \end{equation*}
    % Rewrite this expression as
    % \begin{equation*}
    %     b(X) - (b(Y) - 2 u(equationE(G[Y])) + u(\delta(Y)) - u(E_G(X, Y)))
    % \end{equation*}
    
    We can translate this problem (deciding whether perfect $b$-matching exists)
    to the problem of perfect matching using constructions we used in the
    lectures to prove theorem about $b$-matching polytopes. First we have to
    encode $u$ into vertex bounds, by subdividing the edges and setting $b$ on
    the new middle vertices to be equal to the edge capacity of the edge that we
    subdivided. And secondly we have to somehow get a graph with $b \equiv 1$,
    which we do by replacing every vertex $v$ in this graph by $b(v)$ vertices
    and replace every edge $e = \{u, v\}$ by a full bipartite graph between set
    of vertices that replaced $u$ and set of vertices that replaced $v$.

    Notation:
    Let $G$ be the original graph and $\tilde{G}$ the graph we get after these
    two transformations. We have vertices
    \begin{multline*}
        V(\tilde{G}) = \{ (v, i) \mid v \in V(G), i \in [b(v)] \} \cup \\
        \cup \{ ((e, u), i), ((e, v), i) \mid e = \{u, v\} \in E(G), i \in [u(e)] \}
    \end{multline*}
    and edges 
    \begin{multline*}
        E(\tilde{G}) = \{ \{ (v, i), ((e, v), j) \} \mid v \in V(G), e \in
        \delta(v), i \in [b(v)], j \in [u(e)] \} \cup \\
        \cup \{ \{ ((e, u), i), ((e, v), i) \} \mid e = \{u, v\} \in E(G), i \in [u(e)] \}
    \end{multline*}

    % TODO equivalence of G having a perfect b-matching and \tilde{G} having a
    % perfect matching

    We show that sets $X, Y \subseteq V(G)$ disjoint and violating the property
    described in the exercise exist if and only if there exists a set $Z \subseteq
    \tilde{G}$ violating the Tutte condition (in graph $\tilde{G}$).

    Suppose first that there exist $X, Y \subseteq V(G)$ disjoint which violate
    the condition described in the exercise. We want to define a set $Z
    \subseteq V(\tilde{G})$, for which number of odd components in $G - Z$ will
    be greater than size of $Z$ (failing Tutte condition). The expression in the
    exercise already hints what to take. Namely we take:
    \begin{equation*}
        Z = \{ (v, i) \in \tilde{G} \mid v \in X, i \in [b(v)] \} \cup \{ ((e, v), i) \in
        \tilde{G} \mid v \in Y, e \in \delta(v), i \in [u(e)] \}.
    \end{equation*}
    This amounts to exactly $\sum_{v \in X} b(v) + \sum_{y \in Y} \sum_{e \in
    \delta(y)} u(e)$ vertices. It is not exactly like the expression in the
    exercise, but the difference will be made up by new odd components
    (singletons even), arising from vertices $(v, i) \in \tilde{G}$ for $v \in
    Y$ and from vertices $((e, u), i) \in \tilde{G}$ for $u \in X$.

    By assumption there is an odd number of components $C$ in $G - X - Y$ with
    $\sum_{v \in C} b(v) + \sum_{e \in E_G(V(C), Y)} u(e)$, each of which
    corresponds to an odd connected component in $\tilde{G} - Z$. The component
    $\tilde{C} \subseteq \tilde{G}$ it corresponds to, has
    \begin{equation*}
        \sum_{v \in C} \sum_{i \in [b(v)]} 1 + \sum_{e \in E_G(V(C), Y)} \sum_{i
        \in [u(e)]} 1 + 2 \sum_{e \in E_G(V(C), X)} \sum_{i \in [u(e)]} 1
    \end{equation*}
    vertices.
    The last summand is even, so it does not change the parity.

    But as mentioned above, we actually get more odd components in $\tilde{G}$
    than we had in $G$. Namely, every $(v, i) \in \tilde{G}$ for $v \in Y$ is an
    isolated vertex in $\tilde{G} - Z$, since all $((e, v), i) \in Z$ for $e
    \in \delta(v)$. And, for every $e \in E_G(X, Y)$ we have isolated vertices
    $((e, u), i)$ for $u \in X$, since all $(u, i) \in Z$ and $((e, v), i) \in
    Z$ where $e = \{u, v\}$. Summing all odd components, we get that
    there is strictly more than
    \begin{equation*}
        \sum_{v \in X} b(v) + \sum_{y \in Y} \left( \sum_{e \in \delta(y)} u(e)
        - b(y) \right) - \sum_{e \in E_G(X, y)} u(e) + \sum_{y \in Y} b(y) +
        \sum_{e \in E_G(X, y)} u(e)
    \end{equation*}
    which simplifies to 
    \begin{equation*}
        \sum_{v \in X} b(v) + \sum_{y \in Y} \sum_{e \in \delta(y)} u(e).
    \end{equation*}
    But this is exactly the size of $Z$. So $\tilde{G}$ does not have a perfect
    matching and therefore $G$ does not have a perfect $b$-matching.

    Now assume $G$ does not have a perfect $b$-matching. Therefore $\tilde{G}$
    does not a perfect matching. So by Tutte's theorem we have a set of vertices
    $Z$ that is violating Tutte condition. We will first make a series of
    assumptions about $Z$, effectively modifying it into a specific form, and
    construct $X$ and $Y$ from it.
    %TODO add assumption that Z is maximal in terms of the difference

    If $(v, i) \in Z$, then $\forall j \in [b(v)] \colon (v, j) \in Z$. We can
    assume this, because if $(v, j) \notin Z$, then removing all $(v, i)$ from
    $Z$ does not join any components that were previously disconnected. Though
    it may change an odd component into an even component, that is mitigated by
    removing a vertex from $Z$.

    % We can assume
    % \begin{itemize}
    %     \item{
    %             If $(v, i) \in Z$, then $\forall j \in [b(v)] \colon (v, j) \in
    %             Z$. We can assume this, because if $(v, j) \notin Z$, then
    %             removing all $(v, i)$ from $Z$ does not join any components that
    %             were previously disconnected. Though it may change an odd
    %             component into an even component, that is mitigated by removing
    %             a vertex from $Z$.
    %             %TODO formally
    %         }
    %     \item{
    %             If $e = \{u, v\} \in G$ and $(u, i), (v, i) \in Z$, then we can
    %             assume $((e, v), j), ((e, u), j) \notin Z$ for any $j$. Here we
    %             have a connected component consisting of vertices $\{((e, v),
    %             j), ((e, u), j) \mid j \in [u(e)]\}$ that is a full bipartite
    %             graph and both vertex sets have the same cardinality. This
    %             clearly has a perfect matching, so any set of vertices we would
    %             pick, we couldn't get more components that the cardinality of the
    %             set. So we can assume we do not pick any vertices.
    %         }
    %     \item{
    %             If $e = \{u, v\} \in G$ and $(u, i), (v, i) \notin
    %             Z$ and $((e, u), i) \in Z$
    %         }
    % \end{itemize}

    Then we can define
    \begin{equation*}
        X = \{ v \in V(G) \mid (v, i) \in Z\ \text{for some $i \in [b(v)]$} \}
    \end{equation*}
    and
    \begin{equation*}
        Y = \{ v \in V(G) \mid (v, i)\ \text{isolated vertex in $\tilde{G} -
        Z$ for some $i \in [b(v)]$} \}.
    \end{equation*}
    
    Note that $(v, i) \in \tilde{G}$ (for some $i$) is isolated if and only if
    $(v, j)$ is isolated for all $j$. Clearly both are equivalent to $Z$
    containing all vertices of the form $((e, v), \_) \in \tilde{G}$, where $e
    \in \delta(v) \subseteq G$. And that $(v, i) \in Z$ (for some $i$) if and
    only if $(v, j) \in Z$ for all $j$.

    Now all we have to show is that these $X$ and $Y$ fail the condition
    described in the exercise.

    Let $C$ be a connected component in $\tilde{G} - Z$ with $|C| > 1$.
    
    We can make the following assumptions about form of $C$ (modifying $Z$ a
    little if necessary):

    \begin{itemize}
        \item{By earlier assumption we have that if $(v, i) \in C$ for some $i \in
        [b(v)]$, then $(v, j) \in C$ for all $j \in [b(v)]$.}

        \item{Let $e = \{u, v\} \in E_G(V(C), X)$ (with $v \in X$). Then we can
            assume that $((e, u), i), ((e, v), i) \in Z$ for all $i \in [u(e)]$.
            If not, we could simply remove all of them and not decrease the
            difference $q_{\tilde{G}}(Z) - |Z|$. In this case only interesting
            situation would be when $Z$ would contain $((e, u), i)$ for all $i
            \in [u(e)]$, and we would thus have $((e, v), i)$ be an isolated
            vertex in $\tilde{G} - Z$ for all $i$. But losing $u(e)$ isolated
            vertices is mitigated by removing $u(e)$ elements from $Z$.}

        \item{Let $e = \{u, v\} \in E_G(V(C), Y)$ (with $v \in Y$). Then by
            definition $((e, v), i) \in Z$ for all $i \in [u(e)]$. And we can
            further assume $((e, u), i) \notin Z$ for all $i \in [u(e)]$,
            removing them if necessary. Since $(u, i)$ are not isolated in
            $\tilde{G} - Z$, we do not join any (previously disconnected)
            components, but we may make the component $C$ not be of odd parity
            anymore, which is mitigated by removing a vertex from $Z$.}

        \item{Let $e = \{u, v\} \in E(\tilde{G}[C])$. Then we can assume $((e,
            v), i), ((e, u), i) \notin Z$ for all $i \in [u(e)]$. Clearly
            removing these vertices does not join any (previously disconnected)
            components, since all neighbors of these vertices are contained in
            component $C$, more precisely: $N(((e, v), i)) \setminus Z \subseteq
            C$ and $N(((e, u), i)) \setminus Z \subseteq C$ for all $i$. But
            again, removing vertices $((e, v), i)$ and $((e, u), i)$ may change
            the parity of component, but that is again mitigated by removing a
            vertex from $Z$.}
    
        \item{Let $e = \{u, v\} \in E(G)$ with $(u, i) \in C$ and $(v, i) \in
            C'$ for some other component $C'$ with $|C'| > 1$. Then we can
            remove $((e, v), i), ((e, u), i) \in Z$ for all $i \in [u(e)]$. This
            way we connect two previously distinct components and thus lose at
            most $2$ odd components. If we remove at least $2$ vertices, thats
            no problem. But if there existed was only one $((e, w), i) \in Z$
            ($w \in \{u, v\}$), then after removing $((e, w), i)$ still have $1$
            odd component. So in no cases do we decrease the difference
            $q_{\tilde{G}}(Z) - |Z|$. That means two vertices $u, v \in V(G - X
            - Y)$ are in the same component exactly when $(u, i)$ and $(v, j$
            are in the same component in $\tilde{G} - Z$ for any pair $i,j$.}
    \end{itemize}

    Because of the last assumption, we know that every connected component in
    $G - X - Y$ corresponds to exactly one connected component $C$ in $\tilde{G}
    - Z$ with $|C| > 1$.

    So after these assumptions components $C$ have a much simpler form, while
    still violating Tutte condition in $\tilde{G}$. Let $C$ be a connected
    component with $|C| > 1$ (avoid those components that come from vertices of $Y$
    and vertices from edges between $X$ and $Y$). Then we calculate
    \begin{multline*}
        |V(C)| = \sum_{\substack{v \in V(G)\\ (v, i) \in C}} b(v) + 2
        \sum_{\substack{e = \{u, v\} \in E(G)\\ (u, i), (v, i) \in C}} u(e) + 2
        \sum_{\substack{e = \{u, v\} \in E(G)\\ (u, i) \in C, (v, i) \in Z}}
        u(e) + \\
        + \sum_{\substack{e = \{u, v\} \in E(G)\\ (u, i) \in C, v \in Y}} u(e)
    \end{multline*}
    Therefore component $C$ is odd exactly when $\sum_{\substack{v \in V(G)\\
    (v, i) \in C}} b(v) + \sum_{\substack{e = \{u, v\} \in E(G)\\ (u, i) \in C,
    v \in Y}} u(e)$ is odd.

    Since connected components $C'$ in $G - X - Y$ correspond bijectively to
    connected components $C$ in $\tilde{G} - Z$ with $|C| > 1$.

    So number of connected components in $G - X - Y$ with $\sum_{\substack{v \in V(G)\\
    (v, i) \in C}} b(v) + \sum_{\substack{e = \{u, v\} \in E(G)\\ (u, i) \in C,
    v \in Y}} u(e)$ odd is strictly more than
    \begin{equation*}
        |Z| - \sum_{y \in Y} b(y) - \sum_{e \in E_G(X, Y)} u(e)
    \end{equation*}

    Calculating cardinality of $Z$:
    \begin{equation*}
        |Z| = \sum_{v \in X} b(v) + \sum_{y \in Y} \sum_{e \in \delta(y)} u(e)
    \end{equation*}
    Here we use all above assumptions concerning components.

    Join these two equations and get that number of connected components in $G -
    X - Y$ with $\sum_{\substack{v \in V(G)\\ (v, i) \in C}} b(v) +
    \sum_{\substack{e = \{u, v\} \in E(G)\\ (u, i) \in C, v \in Y}} u(e)$ odd is
    strictly more than
    \begin{equation*}
        \sum_{v \in X} b(v) + \sum_{y \in Y} \sum_{e \in \delta(y)} (u(e) -
        b(y)) - \sum_{e \in E_G(X, Y)} u(e).
    \end{equation*}

    This finished the proof that existance of a perfect $b$-matching is
    equivalent to condition in the exercise being satisfied for every disjoint
    $X, Y \subseteq V(G)$.
    


    % By changing
    % the expression in the exercise we can change it to something that will
    % clearly reveal how to define $Z$.
    % \begin{align}
    %     \sum_{v \in X} b(v) + \sum_{y \in Y} \left(\sum_{e \in \delta(y)} u(e) -
    %     b(y) \right) - \sum_{e \in E_G(X, Y)} u(e) = \\
    %     = \sum_{v \in X} b(v) - \sum_{y \in Y} b(y) \left(\sum_{e \in \delta(y)} u(e) -
    %     b(y) \right) - \sum_{e \in E_G(X, Y)} u(e) = \\
    % \end{align}
\end{exercise}

\end{document}
