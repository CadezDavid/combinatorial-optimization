\newcommand{\sheet}{3}
\documentclass{article}

\usepackage[english, german]{babel}
\usepackage{amsthm,amssymb,amsmath,mathrsfs,mathtools}
\usepackage[shortlabels]{enumitem}
\usepackage{algpseudocode}
\usepackage{hyperref}
\usepackage{tikz}
\usepackage{tikz-cd}

% \usepackage[tmargin=1.25in,bmargin=1.25in,lmargin=1.2in,rmargin=1.2in]{geometry}


\newcommand{\C}{\mathbb{C}}
\newcommand{\R}{\mathbb{R}}
\newcommand{\N}{\mathbb{N}}
\newcommand{\Q}{\mathbb{Q}}
\newcommand{\Z}{\mathbb{Z}}

\DeclareMathOperator{\id}{id}
\DeclareMathOperator{\im}{im}
\DeclareMathOperator{\GL}{GL}
\DeclareMathOperator{\sgn}{sgn}
\DeclareMathOperator{\Tor}{Tor}
\DeclareMathOperator{\Sym}{Sym}
\DeclareMathOperator{\coker}{coker}
\DeclareMathOperator{\Quot}{Quot}
\DeclareMathOperator{\supp}{supp}
\DeclareMathOperator{\Hom}{Hom}
\DeclareMathOperator{\Spec}{Spec}
\DeclareMathOperator{\MinSpec}{MinSpec}
\DeclareMathOperator{\argmin}{argmin}


\newenvironment{exercise}[1] {
  \vspace{0.5cm}
  \noindent \textbf{Exercise~{\sheet}.{#1}.}
} {
  \vspace{0.5cm}
}
\newenvironment{claim} { \par\noindent \textbf{Claim.} } { }
\newenvironment{proof_claim} { \par\noindent \textbf{Proof of claim.} } { \qed{} (of claim) }

\title{Combinatorial optimization\\Exercise sheet \sheet}
\author{Solutions by: Anjana E Jeevanand and David Čadež}

\date{\today}


\begin{document}

\maketitle

\begin{exercise}{2}
    \begin{enumerate}[i)]
        \item{
                On other pages
            % We use the same trick as when we showed that an augmenting path
            % exists exactly when matching is not maximum. Let $N$ be a maximum
            % matching and $M$ some matching. Then look at a graph $G'$ on verices
            % $V(G)$ and edges $M \triangle N$, so $G' = (V(G), M \triangle N)$.
            % In graph $G'$ every vertex has degree at most $2$, so $G'$ is a
            % union of circuits, paths and isolated vertices. Since edges on paths
            % and circuits alternate with regards to either being from $M$ or $N$,
            % the circuits do not contribute more edges from $N$ than $M$. And
            % since every path can contain at most $1$ edge more from $N$ than
            % from $M$, we must have at least $|N| - |M|$ paths. These paths
            % clearly correspond to disjoint $M$-augmenting paths in $G$.
        }
        \item{
                On other pages
            % Still consider this graph $G' = (V(G), M \triangle N)$. The amount of
            % edges in this graph is at most $|N| + |M|$ and there are at least
            % $|N| - |M|$ many disjoint paths. Therefore not all can be longer
            % than $\frac{|N| + |M|}{|N| - |M|}$, which shows the existance of one
            % that is shorter than that.
        }
        \item{
                On other pages
            }
        \item{
                From part (iii) we know that lengths of $P_i$ will not decrease.
                Suppose $j = i + 1$, so they are consecutive. If they were not
                disjoint then, again using part (iii), $|E(P_j)| > |E(P_i)|$.

                Let now $j > i$ and $|E(P_i)| = |E(P_j)|$. Then, since
                lengths in the sequence do not decrease,
                \begin{equation*}
                    |E(P_i)| = |E(P_{i+1})| = \dots = |E(P_j)|.
                \end{equation*}
                Suppose there exists a path $P_k$ for some $k \in  \{i + 1,
                \dots, j\}$ such that $E(P_i), E(P_{i+1}), \dots, E(P_k)$ are
                not pairwise disjoint. Take such $k$ to be minimal. Since
                $E(P_i), E(P_{i+1}), \dots, E(P_{k-1})$ are pairwise disjoint,
                there exists $l \in \{i, \dots, k-1\}$ such that $P_l$ and $P_k$
                are not disjoint. Since we can switch the order of augmentation
                if the $M$-augmenting paths are vertex-disjoint, we can simply
                augment over all paths $P_i, \dots, P_{k-1}$ except on $P_l$.
                Then we are again in the same situation as in part (iii) of this
                exercise and we get $|E(P_k)| > |E(P_l)|$. This is contradiction
                with assumption that such a path $P_k$ exists.
            }
        \item{From part (ii) we know that at each step of the sequence $P_1,
            \dots, P_{\nu(G)}$ we will have a path that is at most of length
            $\frac{\nu(G) + |M|}{\nu(G) - |M|}$ where $|M|$ is the size of the
            matching before augmenting. So if we have done $i$ augmentations up
            to now, the set $|M|$ has size $i$. And thus there exists an
            augmentation path of length at most $\frac{\nu(G) + i}{\nu(G) - i}$.
            Since the lengths of these paths strictly increase, all of the paths
            up to this point must have been at most that length. And assume
            ``worst case'' where all paths after $i$-th will be of pairwise
            different lengths. Also keep in mind that all $M$-augmenting paths
            have odd lengths. The following is then the upper limit for number
            of possible different numbers
            \begin{equation*}
                \left\lfloor \frac{1}{2} \left\lfloor \frac{\nu(G) + i}{\nu(G) -
                i} \right\rfloor \right\rfloor + \nu(G) - i
            \end{equation*}
            Since this holds for any $i \in \{1, \dots, \nu(G)\}$, we should
            look at which $i$ this bound is tightest.
            Write $w := \nu(G)$.
            To find that value it is enough to try to minimize
            \begin{equation*}
                \frac{w + i}{w - i} - 2 i + 2 = \frac{w + i - (2i - 2)(w - i)}{w
                - i} = \frac{w + i - 2wi + 2 i^2 - 2i + 2 w}{w - i}
            \end{equation*}
            We can pretend $i \in \R$ and expression continuous, so we can take
            the derivative. We get
            \begin{equation*}
                \frac{(4i - 1 - 2w)(w - i) + 3w - i - 2wi + 2i^2}{\dots} = 
                \frac{-2 i^2 + 4 i w + 2 w -2 w^2}{\dots}
            \end{equation*}
            Look at the zeros of the numerator. It is a polynomial of 2nd
            degree, so they are
            \begin{equation*}
                i_{1,2} = \frac{2w \pm \sqrt{4 w^2 - 4 (w^2 - w)}}{2} = w \pm
                \sqrt{w}
            \end{equation*}
            Since $i \leq w$ and $i \in \N$, we take an integer close to $w -
            \sqrt{w}$. Lets try $i = \lceil w - \sqrt{w} \rceil$. So the number
            of different integers in the sequence $|E(P_i)|_{i = 1,\dots,
            \nu(G)}$ is at most
            \begin{equation*}
                \left\lfloor \frac{1}{2} \left\lfloor \frac{2 w + \lceil -
                \sqrt{w} \rceil}{\lceil \sqrt{w} \rceil}
                \right\rfloor\right\rfloor + w - \lceil w - \sqrt{w} \rceil
            \end{equation*}
            Write $\sqrt{w} = u + d$, where $u = \lfloor \sqrt{w} \rfloor$. Then
            this simplifies to
            \begin{align*}
                \left\lfloor \frac{1}{2} \left\lfloor \frac{2(u^2 + 2 d u + d^2)
                - u}{u} \right\rfloor\right\rfloor & + u = \\
                &= \left\lfloor \frac{1}{2} \left\lfloor 2 u + 4 d +  \frac{2
                d^2}{u} - 1 \right\rfloor \right\rfloor + u \\
                &= 2u + \left\lfloor \frac{1}{2} \left\lfloor 4 d +  \frac{2
                d^2}{u} \right\rfloor - \frac{1}{2} \right\rfloor \\
                &= 2 \sqrt{w} - 2 d + \left\lfloor \frac{1}{2} \left\lfloor
                4d + \frac{2 d^2}{u} \right\rfloor - \frac{1}{2}
                \right\rfloor \\
                &\leq 2 \sqrt{w} - 2 d + \left\lfloor \frac{1}{2} \left\lfloor 4
                + \frac{2}{u} \right\rfloor - \frac{1}{2} \right\rfloor \\
                &\leq 2 \sqrt{w} - 2 d + 1 \\
                &< 2 \sqrt{w} + 1
            \end{align*}
            Which proves there is less than $2 \sqrt{\nu(G)} + 1$ different
            numbers in the sequence $|E(P_i)|_{i = 1,\dots, \nu(G)}$.
            }
        \item{Denote $G = (A \dot\cup B, E)$.

            First note that every $M$-augmenting path is odd, so it has one
            ending in $A$ and other in $B$. Using that remark the algorithm will
            start at unmatched vertices in $A$ and construct a directed graph
            from which we will be able to extract shortest $M$-augmenting paths.

            The algorithm is out of two parts:
            \begin{enumerate}
                \item{BFS starting from unmatched vertices in $A$ and searching
                    for unmatched vertices in $B$, and thus creating some directed
                    graph}
                \item{Many DFS-es starting at the leaves of a directed graph
                    mentioned above and searching for unmatches vertices in $A$.
                    We will keep track of visited vertices to obtain a family of
                    disjoint shortest $M$-augmenting paths.}
            \end{enumerate}

            Let $\tau$ be a function on vertices $V(G)$ such that $\tau(v) = v$
            for $v \in V(G)$ unmatched and $\tau(v) = u$ if $\{u, v\} \in M$.

            We define the following algorithm:
            \begin{algorithmic}
                \State $I = \text{unmatched vertices}$
                \State $P = \emptyset$ \Comment{used to store beginnings of
                paths}
                \State $\mu \equiv \emptyset$ \Comment{edges of a directed graph, as a
                function}
                \State $\pi \equiv \infty$ \Comment{function on vertices}
                \For{$v \in A \cap I$}
                    \State $\pi(v) = 0$
                \EndFor
                \State $\text{queue}\ q = A \cap I$ \Comment{FIFO queue}
                \State $l_{min} = \infty$
                \While{$q$ not empty \& $\pi(q.\text{top}()) < l_{min} $}
                    \State{$v := q.\text{pop()}$}
                    \If{$v \in A$}
                        \For{$u \in N(v) \setminus \{\tau(v)\}$}
                            \If{$\pi(u) > i + 1$}
                                \State $\mu(u) := \{v\}$
                                \State $\pi(u) := i + 1$
                                \State q.add($u$)
                            \ElsIf{$\pi(u) = i + 1$}
                                \State $\mu(u) := \mu(u) \cup \{v\}$
                            \EndIf
                        \EndFor
                    \ElsIf{$\tau(v) \not= v$}
                        \State $\mu(\tau(v)) := v$
                        \State $\pi(\tau(v)) := i + 1$
                        \State $q$.add($\tau(v)$)
                    \Else
                        \State P.add($v$)
                        \State $l_{min} := \pi(v)$
                    \EndIf
                \EndWhile
                \State \Return{P, $\mu$}

                % \For{$p \in P$}
                %     \State $l_p := \text{length of path } r, \mu(r),
                %     \mu(\mu(r)) \dots \text{(until a vertex repeats)}$
                % \EndFor
                % \State $P := \{ p \mid l_p = \min_s l_s \}$

                % \State $S := \emptyset$
                % \State $R := \emptyset$
                % \State $\text{queue}\ q := P$
                % \State $\varphi := \id_{V(G)}$
                % \While{$q$ not empty}
                %     \State{$v := q.\text{pop()}$}
                %     \If{$\mu(v) = v$} \Comment{meaning $v \in I \cap A$}
                %         \State R.append($v$)
                %     \ElsIf{$\mu(v) \notin S$}
                %         \State $q$.append($\mu(v)$)
                %         \State $\varphi(\mu(v)) := v$
                %         \State S.append($\mu(v)$)
                %     \EndIf
                % \EndWhile

                % \For{$r \in R$}
                %     \State $l_r := \text{length of path } r, \varphi(r),
                %     \varphi(\varphi(r)) \dots \text{(until a vertex repeats)}$
                % \EndFor
                % \State $R := \{ r \mid l_r = \min_s l_s \}$
                % \State \Return{R, $\varphi$}
            \end{algorithmic}

            Description:
            The algorithm is a modified BFS. It keeps stored:
            \begin{itemize}
                \item{distance (length of shortest $M$-alternating path) of each
                    visited element to the nearest element of $I \cap A$}
                \item{directed edges going along shortest $M$-alternating paths
                    towards elements of $I \cap A$}
            \end{itemize}
            At the end it returns $P$, which are the elements of $I \cap B$ that
            are in the union of shortest $M$-augmenting paths. If $P$ is empty,
            then $I \cap B$ is not reachable from any vertex in $I \cap A$ with
            $M$-augmenting paths. This means matching $M$ is maximum matching
            (using Berge theorem).

            The time complexity of this algorithm is clearly $\mathcal{O}(m +
            n)$, because BFS is $\mathcal{O}(m + n)$ and we do only a few
            constant operations at each vertex.


            Denote the directed graph $\overline{G} = (V(G), \{ v \rightarrow
            u \mid v \in V(G), u \in \mu(v) \} )$. This directed graph is
            clearly acyclic since for each edge $v \rightarrow u$ we have
            \begin{equation}\label{graph_property}
                \pi(v) = \pi(u) + 1.
            \end{equation}

            All paths in $\overline{G}$ starting in an edge $v \in P$ and ending
            in an edge $u \in I \cap A$ have the same length, which follows
            from~\ref{graph_property}.

            Note that two paths in $\overline{G}$ are vertex-disjoint exactly
            when their endpoints are different and they are edge-disjoint. This
            follows from the fact that a path is $M$-alternating and if they
            share a vertex, they must also share the edge matching this vertex
            (except endpoints, which we assumed separately). Using this remark
            it is enough to remove edge on that paths in the algorithm.

            On graph $\overline{G}$ we run DFS multiple times. In each iteration
            we find an (shortest) $M$-augmenting path and remove all vertices
            from the graph before next iteration. That way we find some maximal
            family of $M$-augmenting paths from $I \cap B$ to $I \cap A$.

            Sketch of an algorithm:
            \begin{algorithmic}
                \State $\mathcal{P} = \emptyset$
                \For{$v \in P$}
                    \State Run DFS in $\overline{G}$ starting in $v$
                    and searching for any element in $I \cap A$.
                    \If{path was found from $P$ to $I \cap A$}
                        \State Add path to $\mathcal{P}$
                        \State Remove all vertices in $\overline{G}$ that lie on that path
                    \EndIf
                \EndFor
            \end{algorithmic}

            Time complexity of this algorithm is also clearly $\mathcal{O}(m +
            n)$ because DFS is. If we run both parts one after the other the
            time complexity is also $\mathcal{O}(m + n)$.

            Lets prove this family $\mathcal{P}$ satisfies the condition in the
            exercise. Suppose we augmented $M$ using all paths in $\mathcal{P}$.
            Denote the augmented matching $M'$. Since all disjoint paths are
            disjoint, it does not matter in which order we do the augmentating.
            Suppose there exists an $M'$-augmenting path $P'$ for which $|E(P')|
            \leq l_{min}$. Separate cases
            \begin{itemize}
                \item{If it shared a vertex with some path $P \in If
                    \mathcal{P}$, we can use part 3 of this exercise to prove on
                    matching $M \triangle (\cup \mathcal{P} \setminus \{P\})$
                    that $|E(P')| > |E(P)| = l_{min}$.}
                \item{If it is disjoint with all paths in the family
                    $\mathcal{P}$, then $P'$ would be contained in the graph
                    $\overline{G}$. Let $v' \in P' \cap I \cap B$. Then in the
                    for loop when we pick $v'$ we would have found an
                    $M$-augmenting path that starts in $v'$. Therefore
                    $\mathcal{P}$ would contain a path that is not disjoint with
                    $P'$.}
            \end{itemize}
            Therefore such a path cannot exist and thus $\mathcal{P}$ satisfies
            inequality in the exercise.

            % Algorithm is a modified BFS (ran twice), with the condition that a
            % paths must be $M$-alternating. That is reflected in the fact that
            % after arriving in $B$ along an unmatched edge, we have either come
            % to an unmatched vertex in $B$ (in which case we found a path) or our
            % next step will be along the matching edge (going back to $A$).

            % Important note: This algorithm does not stop after finding first $I
            % \cap A \rightarrow I \cap B$ path (it shouldn't stop at the
            % \emph{first} one anyway). It continues running until all vertices
            % have been seen. We could implement some sort of index which we would
            % carry around and increase along the path. Then, when we would find
            % the first unmatched edge in $B$ we would remember the value (meaning
            % the length of the path) and then stop the algorithm after top
            % element in $q$ is further from $A$ than that value. That would for
            % sure speed up the algorithm, but I think it would not change the
            % complexity, so for this exercise this is not necessary. In the worst
            % case we still go over all vertices.

            % After first while loop finishes we get a set $P \subseteq B \cap I$
            % and a function $\mu$. The set $P$ contains the endings of
            % $M$-augmenting paths we are looking for.

            % At this point all we are interested in can be described as a
            % directed subgraph that is an $M$-alternating forest with roots $I
            % \cap A$. All edges are directed towards the root. Note that not
            % necessarily all leaves are elements of $I \cap B$ (most probably
            % aren't), $P$ describes exactly those leaves that are in $I \cap B$.
            % In the next while loop we will start at those leaves that are in $I
            % \cap B$ and go towards the roots. That way we will pick a single
            % path out of every tree in the forest (that contains a vertex from
            % $P$). At the end we still have to compare those paths and pick all
            % shortest ones.

            % So we empty the set $S$, which remembers seen vertices, and put $P$
            % in the queue $q$. Then we do BFS and search towards the root. If two
            % paths of equal length collide in some vertex while going towards the
            % root, we just pick the path that we looked at first (this depends on
            % the order in which we put $P$ into $q$), because it does not matter
            % which one we choose (because they are of the equal length). While
            % going searching towards the root we use $\varphi$ to store path
            % back, towards the leaves in $P$. 

            % At the end we get $\varphi$ and $R$. Each element $r \in R$ defines
            % a path $r, \varphi(r), \varphi(\varphi(r)), \dots$ which leads
            % towards an element in $P$, so towards an unmatched vertex in $B$.
            % Then just go over all of these paths and note their lengths. Then
            % pick all the ones whole length is equal to minimum of their lengths.
            % This gives us a family of paths $\mathcal{P}$, for which the
            % statement in the exercise is true.

            % Suppose there exists an $M'$-augmenting path $V$, for which $|E(V)|
            % \leq l$ where $l$ is the length of paths in $\mathcal{P}$. If 

            }
        \item{Using previous parts of the exercise we can construct an
            algorithm that finds a maximum matching of a bipartite graph in time
            complexity $\mathcal{O}(\sqrt{n}(m + n))$.

            We actually just use the algorithm described above.
            Specificaly:
            \begin{itemize}
                \item{Run algorithm above and get disjoint shortest
                    $M$-augmenting paths.}
                \item{Augment over those paths.}
                \item{Go back to first step.}
            \end{itemize}
            From part (v) of this exercise we know we will have to iterate at
            most $2 \sqrt{\nu(G)} + 1$ times. Using $2 \nu(G) \leq n$ we quickly
            get that all together would be in the class $\mathcal{O}(\sqrt{n}(n
            + m))$.}
    \end{enumerate}
\end{exercise}

\end{document}
