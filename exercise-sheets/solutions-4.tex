\newcommand{\sheet}{4}
\documentclass{article}

\usepackage[english, german]{babel}
\usepackage{amsthm,amssymb,amsmath,mathrsfs,mathtools}
\usepackage[shortlabels]{enumitem}
\usepackage{algpseudocode}
\usepackage{hyperref}
\usepackage{tikz}
\usepackage{tikz-cd}

% \usepackage[tmargin=1.25in,bmargin=1.25in,lmargin=1.2in,rmargin=1.2in]{geometry}


\newcommand{\C}{\mathbb{C}}
\newcommand{\R}{\mathbb{R}}
\newcommand{\N}{\mathbb{N}}
\newcommand{\Q}{\mathbb{Q}}
\newcommand{\Z}{\mathbb{Z}}

\DeclareMathOperator{\id}{id}
\DeclareMathOperator{\im}{im}
\DeclareMathOperator{\GL}{GL}
\DeclareMathOperator{\sgn}{sgn}
\DeclareMathOperator{\Tor}{Tor}
\DeclareMathOperator{\Sym}{Sym}
\DeclareMathOperator{\coker}{coker}
\DeclareMathOperator{\Quot}{Quot}
\DeclareMathOperator{\supp}{supp}
\DeclareMathOperator{\Hom}{Hom}
\DeclareMathOperator{\Spec}{Spec}
\DeclareMathOperator{\MinSpec}{MinSpec}
\DeclareMathOperator{\argmin}{argmin}


\newenvironment{exercise}[1] {
  \vspace{0.5cm}
  \noindent \textbf{Exercise~{\sheet}.{#1}.}
} {
  \vspace{0.5cm}
}
\newenvironment{claim} { \par\noindent \textbf{Claim.} } { }
\newenvironment{proof_claim} { \par\noindent \textbf{Proof of claim.} } { \qed{} (of claim) }

\title{Combinatorial optimization\\Exercise sheet \sheet}
\author{Solutions by: Anjana E Jeevanand and David Čadež}

\date{\today}


\begin{document}

\maketitle

\begin{exercise}{1}
    By greedily and heuristically looking at the graph we find the following
    matching $M = \{ \{4, 15\}, \{1, 5\}, \{2, 6\}, \{7, 8\}, \{3, 10\}, \{9,
    12\}, \{11, 16\} \}$. It has size $7$, so we can try to show $\nu(G) = 7$.
    We can do that by finding a set $X$ such that $q_G(X) = |X| + 2$. Again
    heuristically looking at the graph yields $X = \{1, 16, 8\}$, for which the
    graph $G \setminus X$ contains $5$ odd components, namely $\{5\}$, $\{13\}$,
    $\{11\}$, $\{2, 6, 7\}$, $\{3, 9, 10, 12, 14\}$. So $\max_{X \subseteq V(G)}
    (q_G(X) - |X|) \geq 2$. But also, using Berge-Tutte formula and $\nu(G) \geq
    7$, we have $\max_{X \subseteq V(G)} (q_G(X) - |X|) \leq 2$. So $\nu(G) = 7$
    and $M$ is maximum.
\end{exercise}

\end{document}
