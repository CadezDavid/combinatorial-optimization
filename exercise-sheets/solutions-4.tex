\newcommand{\sheet}{4}
\documentclass{article}

\usepackage[english, german]{babel}
\usepackage{amsthm,amssymb,amsmath,mathrsfs,mathtools}
\usepackage[shortlabels]{enumitem}
\usepackage{algpseudocode}
\usepackage{hyperref}
\usepackage{tikz}
\usepackage{tikz-cd}

% \usepackage[tmargin=1.25in,bmargin=1.25in,lmargin=1.2in,rmargin=1.2in]{geometry}


\newcommand{\C}{\mathbb{C}}
\newcommand{\R}{\mathbb{R}}
\newcommand{\N}{\mathbb{N}}
\newcommand{\Q}{\mathbb{Q}}
\newcommand{\Z}{\mathbb{Z}}

\DeclareMathOperator{\id}{id}
\DeclareMathOperator{\im}{im}
\DeclareMathOperator{\GL}{GL}
\DeclareMathOperator{\sgn}{sgn}
\DeclareMathOperator{\Tor}{Tor}
\DeclareMathOperator{\Sym}{Sym}
\DeclareMathOperator{\coker}{coker}
\DeclareMathOperator{\Quot}{Quot}
\DeclareMathOperator{\supp}{supp}
\DeclareMathOperator{\Hom}{Hom}
\DeclareMathOperator{\Spec}{Spec}
\DeclareMathOperator{\MinSpec}{MinSpec}
\DeclareMathOperator{\argmin}{argmin}


\newenvironment{exercise}[1] {
  \vspace{0.5cm}
  \noindent \textbf{Exercise~{\sheet}.{#1}.}
} {
  \vspace{0.5cm}
}
\newenvironment{claim} { \par\noindent \textbf{Claim.} } { }
\newenvironment{proof_claim} { \par\noindent \textbf{Proof of claim.} } { \qed{} (of claim) }

\title{Combinatorial optimization\\Exercise sheet \sheet}
\author{Solutions by: Anjana E Jeevanand and David Čadež}

\date{\today}


\begin{document}

\maketitle

\begin{exercise}{1}
    By greedily and heuristically looking at the graph we find the following
    matching $M = \{ \{4, 15\}, \{1, 5\}, \{2, 6\}, \{7, 8\}, \{3, 10\}, \{9,
    12\}, \{11, 16\} \}$. It has size $7$, so we can try to show $\nu(G) = 7$.
    We can do that by finding a set $X$ such that $q_G(X) = |X| + 2$. Again
    heuristically looking at the graph yields $X = \{1, 16, 8\}$, for which the
    graph $G \setminus X$ contains $5$ odd components, namely $\{5\}$, $\{13\}$,
    $\{11\}$, $\{2, 6, 7\}$, $\{3, 9, 10, 12, 14\}$. So $\max_{X \subseteq V(G)}
    (q_G(X) - |X|) \geq 2$. But also, using Berge-Tutte formula and $\nu(G) \geq
    7$, we have $\max_{X \subseteq V(G)} (q_G(X) - |X|) \leq 2$. So $\nu(G) = 7$
    and $M$ is maximum.
\end{exercise}

\begin{exercise}{2}
    I assume the exercise means we can use that fact that there exists and
    algorithm that finds maximal family of disjoint $M$-augmenting paths (in a
    general graph) and we do not have to make it.

    So let $\epsilon > 0$ and $G$ an indirected graph. WLOG assume
    $\frac{1}{\epsilon} = k \in \N$.

    Start with $M = \emptyset$ and use the algorithm on $G$ and $M$ to find a
    family $\mathcal{P} = \{ P_1, \dots, P_k \}$ of augmenting paths, such that
    after augmenting the matching, there are no more augmenting paths of that
    length in $G$. So in every loop, we strictly increasy the legths of path in
    $\mathcal{P}$ by at least $2$.

    Repeat this loop $k$ times. The length of paths in the last step was at
    least $2 k - 1$. So now there is no $M$-augmenting paths of length $2 k - 1$
    in $G$ anymore (all, if any, are strictly longer).

    Time complexity of this scheme is $\mathcal{O}(k (m + n))$ because we simply
    run a $\mathcal{O}(m + n)$ algorithm $k$ times.

    Now all that is left, is to prove this matching $M$ is suficiently big, ie
    that $\frac{|M|}{\nu(G)} \geq 1 - \epsilon$. Indeed, take any maximum
    matching $N$ in $G$ and look at the graph $G' = (V(G), M \triangle N)$.
    Every cycle in $G'$ contains same amount of edges from $M$ as from $N$. And
    every path, say of length $2 l + 1$, contains $l$ edges from $M$ and $l + 1$
    from $N$, meaning the ratio $\frac{\text{\# edges from}\ M}{\text{\# edges
    from}\ N}$ is $\frac{l}{l+1}$. Using that paths in $G'$ are at least of
    length $2 k + 1$, we get that the ratio is at least $\frac{k}{k + 1}$. The
    fraction $\frac{|M \setminus N|}{|N \setminus M|}$ is then clearly at least
    $\frac{k}{k + 1}$ (since it is at least that much on every connected
    component). Then
    \begin{equation*}
        \frac{k}{k + 1} \leq
        \frac{|M \setminus N|}{|N \setminus M|} \leq
        \frac{|M| - |M \cap N|}{|N| - |M \cap N|} \leq 
        \frac{|M|}{|N|}
    \end{equation*}
    So we have shown that $|M| \geq (1 - \frac{1}{k + 1}) \nu(G) \geq (1 -
    \epsilon) \nu(G)$, which we had to show.
\end{exercise}

\end{document}
