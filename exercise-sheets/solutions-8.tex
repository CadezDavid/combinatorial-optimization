\newcommand{\sheet}{8}
\documentclass{article}

\usepackage[english, german]{babel}
\usepackage{amsthm,amssymb,amsmath,mathrsfs,mathtools}
\usepackage[shortlabels]{enumitem}
\usepackage{algpseudocode}
\usepackage{hyperref}
\usepackage{tikz}
\usepackage{tikz-cd}

% \usepackage[tmargin=1.25in,bmargin=1.25in,lmargin=1.2in,rmargin=1.2in]{geometry}


\newcommand{\C}{\mathbb{C}}
\newcommand{\R}{\mathbb{R}}
\newcommand{\N}{\mathbb{N}}
\newcommand{\Q}{\mathbb{Q}}
\newcommand{\Z}{\mathbb{Z}}

\DeclareMathOperator{\id}{id}
\DeclareMathOperator{\im}{im}
\DeclareMathOperator{\GL}{GL}
\DeclareMathOperator{\sgn}{sgn}
\DeclareMathOperator{\Tor}{Tor}
\DeclareMathOperator{\Sym}{Sym}
\DeclareMathOperator{\coker}{coker}
\DeclareMathOperator{\Quot}{Quot}
\DeclareMathOperator{\supp}{supp}
\DeclareMathOperator{\Hom}{Hom}
\DeclareMathOperator{\Spec}{Spec}
\DeclareMathOperator{\MinSpec}{MinSpec}
\DeclareMathOperator{\argmin}{argmin}


\newenvironment{exercise}[1] {
  \vspace{0.5cm}
  \noindent \textbf{Exercise~{\sheet}.{#1}.}
} {
  \vspace{0.5cm}
}
\newenvironment{claim} { \par\noindent \textbf{Claim.} } { }
\newenvironment{proof_claim} { \par\noindent \textbf{Proof of claim.} } { \qed{} (of claim) }

\title{Combinatorial optimization\\Exercise sheet \sheet}
\author{Solutions by: Anjana E Jeevanand and David Čadež}

\date{\today}


\begin{document}

\maketitle

\begin{exercise}{1}
    The algorithm is very similar to Christofides' Algorithm.
    \begin{algorithm}
        \caption{Metric $s$-$t$ path TSP}
        \textbf{Input} $(K_n, c)$ a metric TSP instance and two vertices $s, t
        \in V(K_n)$. \\
        \textbf{Output} A Hamiltonian $s$-$t$ path of minimum weight.
        \begin{algorithmic}[1]
            \STATE{Compute MST T for $(K_n, c)$.}
            \STATE{Let $W := \text{odd}(T) \triangle \{s, t\}$ be the set of odd
                degree vertices in $T$, while adding $s$ if degree of $s$ in $T$ is
                even and removing it if it is odd, and same for $t$.}
            \STATE{Compute minimum weight $W$-join $J$.}
            \STATE{Compute Eulerian $s$-$t$ path in $E(T) \cup J$. We can do
            this because all vertices have even degree with respect to $E(T)
            \cup J$, except $s$ and $t$.}
            \STATE{Define $\overline{T}$ an $s$-$t$ path in Eulerian path above. More
                specifically: add them in order of first appearance, except $t$,
                which should be added last.}
        \end{algorithmic}
    \end{algorithm}

    Now we show $c(\overline{T}) \leq \frac{5}{3} c(\text{OPT})$, where
    $\text{OPT}$ is minimum weight Hamiltonian $s$-$t$ path in $(K_n, c)$.

    Since $\text{OPT}$ is a tree, we have $c(T) \leq c(\text{OPT})$.

    Now we want to show $c(J) \leq \frac{2}{3} c(\text{OPT})$.
    It suffices to show $3 c(J) \leq c(T) + c(\text{OPT})$.
    We want to show we can find $3$ disjoint $W$-joins in the disjoint union
    $E(T) \sqcup E(\text{OPT})$.

    For every vertex $w \in W$ we have $\deg_{E(T) \sqcup
    E(\text{OPT})}(w) \geq 3$ and odd:\\
    If $w \notin \{s, t\}$, then $\deg_{E(\text{OPT})}(w) = 2$ and
    $\deg_{E(T)}(w) \geq 1$ odd.\\
    If $w \in \{s, t\}$, then $\deg_{E(\text{OPT})}(w) = 1$. By definition of
    $W$ we have $w \notin \text{odd}(T)$, so also $\deg_{E(T)}(w) \geq 2$ even.

    First we construct one $W$-join along the path $\text{OPT}$.
    This we can do by ordering $W = \{w_1, \dots, w_{2k}\}$ in the order in
    which they appear on the path $\text{OPT}$. Then simply take segments from
    $w_{2i - 1}$ to $w_{2i}$ for $i = 1, \dots k$.

    We can now remove these segments from $E(T) \sqcup E(\text{OPT})$, to obtain
    $E'$.
    By arguments above, every vertex in $W$ now has even degree $\geq 2$ with
    respect to $E'$. This means there exists a Eulerian tour in in $E'$, which
    we can split into to $W$-joins by taking alternating segments.

    Each of these three joins has cost greater or equal to $c(J)$, so $3c(J)
    \leq c(\text{OPT}) + c(T) \leq 2 c(\text{OPT})$.
    So
    \begin{equation*}
        c(\overline{T}) \leq c(T) + c(J) \leq c(\text{OPT}) + \frac{2}{3}
        c(\text{OPT}) \leq \frac{5}{3} c(\text{OPT})
    \end{equation*}
\end{exercise}

\begin{exercise}{2}
    We can show this by using formulation in exercise $4$.

    %TODO what if u is infinite and thus P not necessarily bounded
    If bounds $u$ are finite, then $P$ is bounded. It is also clearly closed, so
    it is compact.

    Condition (i): if $0 \leq x \leq y \in P$, we have $s$-$t$ flow $f$ with
    $f(e) = y_e$ for all $e \in U$. We can reduce this flow $f$ along $s$-$t$
    paths to obtain $f(e) = x_e$ on each $e \in U$. So $x \in P$.

    Condition (ii): let $x \in \R^U_+$ with $y, z \in P$ and $y, z \leq x$, such
    that $y$ and $z$ are maximal such. We have to show that flows $f$ and $g$,
    corresponding to $y$ and $z$ respectively are maximum flows. Suppose there
    excists an augmenting $s$-$t$ path for either flow. Since its a path, it
    does not return to $s$, and thus it can only contain one edge $e \in U$. But
    this violates maximality of $y$ and $z$. Therefore $f$ and $g$ are both
    maximum flows and thus $\mathbb{1} y = \mathbb{1} z$.
\end{exercise}

\end{document}
