\newcommand{\sheet}{6}
\documentclass{article}

\usepackage[english, german]{babel}
\usepackage{amsthm,amssymb,amsmath,mathrsfs,mathtools}
\usepackage[shortlabels]{enumitem}
\usepackage{algpseudocode}
\usepackage{hyperref}
\usepackage{tikz}
\usepackage{tikz-cd}

% \usepackage[tmargin=1.25in,bmargin=1.25in,lmargin=1.2in,rmargin=1.2in]{geometry}


\newcommand{\C}{\mathbb{C}}
\newcommand{\R}{\mathbb{R}}
\newcommand{\N}{\mathbb{N}}
\newcommand{\Q}{\mathbb{Q}}
\newcommand{\Z}{\mathbb{Z}}

\DeclareMathOperator{\id}{id}
\DeclareMathOperator{\im}{im}
\DeclareMathOperator{\GL}{GL}
\DeclareMathOperator{\sgn}{sgn}
\DeclareMathOperator{\Tor}{Tor}
\DeclareMathOperator{\Sym}{Sym}
\DeclareMathOperator{\coker}{coker}
\DeclareMathOperator{\Quot}{Quot}
\DeclareMathOperator{\supp}{supp}
\DeclareMathOperator{\Hom}{Hom}
\DeclareMathOperator{\Spec}{Spec}
\DeclareMathOperator{\MinSpec}{MinSpec}
\DeclareMathOperator{\argmin}{argmin}


\newenvironment{exercise}[1] {
  \vspace{0.5cm}
  \noindent \textbf{Exercise~{\sheet}.{#1}.}
} {
  \vspace{0.5cm}
}
\newenvironment{claim} { \par\noindent \textbf{Claim.} } { }
\newenvironment{proof_claim} { \par\noindent \textbf{Proof of claim.} } { \qed{} (of claim) }

\title{Combinatorial optimization\\Exercise sheet \sheet}
\author{Solutions by: Anjana E Jeevanand and David Čadež}

\date{\today}


\begin{document}

\maketitle

\begin{exercise}{2}
    We have an undirected graph $G$ and $T \subseteq V(G)$, with $|T| = 2k$.

    We have to show that minimum cardinality $T$-cut in $G$ equals maximum of
    $\min_i \lambda_{s_i, t_i}$ over pairings $T = \{s_1, t_1, \dots, s_k,
    t_k\}$ where $\lambda_{s, t}$ denotes maximum number of pairwise
    edge-disjoint $s$-$t$-paths.

    First show that every $T$-cut will be bigger than $\min_i \lambda_{s_i,
    t_i}$ for any pairing. Take a pairing $T = \{s_1, t_1, \dots, s_k, t_k\}$
    and a $T$-cut $C = \delta(X)$. The cut $C$ has separate at least one pair
    (say $\{s_j, t_j\}$), otherwise $|X \cap T|$ would be even. And since there
    are at least $\min_i \lambda_{s_i, t_i}$ edge-disjoint $s_j$-$t_j$-paths, we
    must have $|C| \geq \min_i \lambda_{s_i, t_i}$.

    This gives us inequality that minimum cardinality of a $T$-cut is greater or
    equal to the maximum of $\min_i \lambda_{s_i, t_i}$ over pairings $T$.

    Now we have to show that this inequality is in fact equality.

    Remember (from previous courses) that for vertices $s, t \in v(G)$,
    the number of pariswise edge-disjoint $s$-$t$-paths is equal to the
    cardinality of a minimum $s$-$t$-cut.
    
    And for computing $s$-$t$-cuts we have Gomory-Hu trees, so let $u \equiv 1$
    and let $H$ be a Gomory-Hu tree for $(G, u)$. Then $\lambda_{s, t} = \min_{e
    \in P_{s, t}} u(e)$, where $P_{s, t}$ is the (unique) $s$-$t$-path in $H$.

    Then we use a theorem from the lectures, which stated that minimum capacity
    $T$-cut can be found among fundamental ones in Gomory-Hu tree.

    Define a subset of edges $F = \triangle_{s, t \in T, s\not=t} P_{s, t}$, a symmetric
    difference over (unique) $s$-$t$-paths in Gomory-Hu tree over all pairs
    $\{s, t\} \subseteq T$ (at this point $T$ is just a set, not a pairing).

    \begin{claim}
        For every edge in $e \in H$, the cut at edge $e$ is a $T$-cut if and
        only if $e \in F$.
    \end{claim}
    \begin{proof_claim}
        Let $e \in H$.
        Removing an edge $e$, the tree $H$ splits into two components, say $C_1$
        and $C_2$. Let $|C_1 \cap T| = p$ and $|C_2 \cap T| = r$. Since $2k =
        p + r$, $p$ and $r$ have the same parity. Observe that $e$ lies exactly
        on $pr$ paths, exactly on those, for which elements of the pair come
        from different components.

        Therefore: $e \in F$ $\Leftrightarrow$ $pr$ odd $\Leftrightarrow$ $p$
        odd $\Leftrightarrow$ $e$ defines a $T$-cut.
    \end{proof_claim}

    So minimum cardinality $T$-cut can be found in $F$. Now we just have to find
    a pairing, such that all paths will be contained $F$. This also follows from
    the claim above: an edge $e \in E(H) \setminus F$ always splits the tree
    into two components, each of which contains even number of vertices from
    $T$. We can then use the claim on components and keep on removing edges in
    $E(H) \setminus F$, at each step all components having even number of
    elements from $T$.

    Now take a pairing so that for each pair both elements lie in the same
    connected component of $(H, F)$. The paths will all lie in $F$, therefore
    \begin{equation*}
        \min_i \lambda_{s_i, t_i} \geq \min_{e \in F} u(e) = \text{min.
        cardinality $T$-cut}.
    \end{equation*}
\end{exercise}

\begin{exercise}{3}
    Let $R = V(G) \setminus (S_e \cup S_o)$.

    First consider the existence of a solution. A solution exists exactly when
    there exists some $T$ with $S_o \subseteq T \subseteq S_o \cup R$ such that
    there exists a $T$-join in $G$. Existence of a $T$-join is equivalent (as we
    showed in the lectures) to each connected component of $G$ containing even
    number of vertices from $T$. Putting these two together, we get that a
    solution exists exactly when for every connected component $C$ of $G$ one of
    the following holds
    \begin{itemize}
        \item{$|C \cap S_o|$ odd, or}
        \item{$|C \cap S_o|$ even and $|C \cap R| > 0$.}
    \end{itemize}
    In the first case we have a $(C \cap S_o)$-join in component $C$ and in the
    second case we have a $((C \cap S_o) \cup \{r\})$-join, where $r \in C \cap
    R$ any.

    Next we contract $R$ into a single vertex, call it $r$, and define edge weights in the
    following way:

    \[
        c'(e) = \begin{cases}
            \min_{u' \in R} c(\{u', v\})  & \text{if $e = \{u, v\} \in \delta(r)$
            (WLOG $u = r$)} \\
            c(e) & \text{else}
        \end{cases}
    \]

    Call this graph $G'$.

    Then, if $|S_o|$ is odd, solve MINIMUM WEIGHT $(S_o \cup \{r\})$-JOIN
    PROBLEM in graph $G'$. Otherwise solve MINIMUM WEIGHT $S_o$-JOIN PROBLEM in
    graph $G'$.

    Let $J' \subseteq E(G')$ be a solution to any one of the above problems.
    Define $J$ in the following way:

    Add every edge $e \in J'$, with $e \notin \delta(r)$, to $J$. For every edge
    $e = \{r, v\}$ add an edge $\{u, v\}$ to $J$, where $u$ is one of
    $\argmin_{u' \in R} c(\{u', v\})$.


\end{exercise}

\end{document}
