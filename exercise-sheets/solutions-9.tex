\newcommand{\sheet}{9}
\documentclass{article}

\usepackage[english, german]{babel}
\usepackage{amsthm,amssymb,amsmath,mathrsfs,mathtools}
\usepackage[shortlabels]{enumitem}
\usepackage{algpseudocode}
\usepackage{hyperref}
\usepackage{tikz}
\usepackage{tikz-cd}

% \usepackage[tmargin=1.25in,bmargin=1.25in,lmargin=1.2in,rmargin=1.2in]{geometry}


\newcommand{\C}{\mathbb{C}}
\newcommand{\R}{\mathbb{R}}
\newcommand{\N}{\mathbb{N}}
\newcommand{\Q}{\mathbb{Q}}
\newcommand{\Z}{\mathbb{Z}}

\DeclareMathOperator{\id}{id}
\DeclareMathOperator{\im}{im}
\DeclareMathOperator{\GL}{GL}
\DeclareMathOperator{\sgn}{sgn}
\DeclareMathOperator{\Tor}{Tor}
\DeclareMathOperator{\Sym}{Sym}
\DeclareMathOperator{\coker}{coker}
\DeclareMathOperator{\Quot}{Quot}
\DeclareMathOperator{\supp}{supp}
\DeclareMathOperator{\Hom}{Hom}
\DeclareMathOperator{\Spec}{Spec}
\DeclareMathOperator{\MinSpec}{MinSpec}
\DeclareMathOperator{\argmin}{argmin}


\newenvironment{exercise}[1] {
  \vspace{0.5cm}
  \noindent \textbf{Exercise~{\sheet}.{#1}.}
} {
  \vspace{0.5cm}
}
\newenvironment{claim} { \par\noindent \textbf{Claim.} } { }
\newenvironment{proof_claim} { \par\noindent \textbf{Proof of claim.} } { \qed{} (of claim) }

\title{Combinatorial optimization\\Exercise sheet \sheet}
\author{Solutions by: Anjana E Jeevanand and David Čadež}

\date{\today}


\begin{document}

\maketitle

\begin{exercise}{1}
    
    Let $U$ be a finite set. Define a condition for $f \colon 2^U \to \R$ that
    \begin{equation}\label{condition}
        f(X \cup \{y, z\}) - f(X \cup \{y\}) \leq f(X \cup \{z\}) - f(X)
    \end{equation}
    for every $X \subseteq U$ and $y, z \in U$ with $y \not= z$.

    If $f$ is submodular, then condition~\ref{condition} follows from the
    definition of submodularity by setting one set to be $X \cup \{y\}$ and
    other $X \cup \{z\}$.

    Suppose now $f$ satisfies condition~\ref{condition}.
    Take $X, Y \subseteq U$. We are trying to show
    \begin{equation*}
        f(X \cup Y) + f(X \cap Y) \leq f(X) + f(Y).
    \end{equation*}
    We do induction on $n = |X \setminus Y| + |Y \setminus X|$.
    % Observe that 
    % \begin{equation*}
    %     f(X \cup \{y_1, y_2, z\}) - f(X \cup \{y_1, y_2\}) \leq f(X \cup \{y_1,
    %     z\}) - f(X \cup \{y_1\}) \leq f(X \cup \{z\}) - f(X)
    % \end{equation*}
    % which demonstrates that 
    % \begin{equation*}
    %     f(X \cup Y \cup \{z\}) - f(X \cup Y) \leq f(X \cup \{z\}) - f(X)
    % \end{equation*}
    % for any $Y \subseteq U$ with $z \notin Y$.

    If $X \subseteq Y$ or $Y \subseteq X$, then the statement holds.

    Assume now that the statement holds for $k < |X \setminus Y| + |Y \setminus
    X| = n$.
    Take $x \in X \setminus Y$. By induction hypothesis we have
    \begin{equation}\label{ineq1}
        f((X \setminus \{x\}) \cup Y) + f(X \cap Y) \leq f(X \setminus \{x\}) + f(Y).
    \end{equation}

    By condition~\ref{condition} we also have the following chain of
    inequalities
    \begin{equation}\label{ineq2}
    \begin{split}
        f(X \cup Y) - f((X \setminus \{x\}) \cup Y) &\leq f(X \cup Y_{n-1}) -
        f((X \cup Y_{n-1} \setminus \{x\}) \\
        &\leq \dots \\
        &\leq f(X \cup Y_2) - f((X \cup Y_2) \setminus \{x\}) \\
        &\leq f(X \cup Y_1) - f((X \cup Y_1) \setminus \{x\}) \\
        &\leq f(X) - f(X \setminus \{x\}).
    \end{split}
    \end{equation}
    where $Y = \{y_1, \dots, y_n\}$ and $Y_i = \{y_1, \dots, y_i\}$.

    Summing~\ref{ineq1} and~\ref{ineq2} yields
    \begin{equation*}
        f(X \cup Y) + f(X \cap Y) \leq f(X) + f(Y),
    \end{equation*}
    which is what we wanted to show.
\end{exercise}

\begin{exercise}{3}
    Let $B_f$ denote the base polyhedron of $f$.

    Take some total order $\prec$ of $U$. We show that $b^\prec$ is a
    vertex of $U$.

    First we show that $b^\prec \in B_f$.
    Take some $A \subseteq U$.
    By definition
    \begin{equation*}
        b^\prec (A) = \sum_{a \in A} f(U_{\preceq a}) - f(U_{\preceq a}).
    \end{equation*}
    From the first exercise it follows that
    \begin{equation*}
        f(U_{\preceq a}) - f(U_{\preceq a}) \leq f(A_{\preceq a}) - f(A_{\preceq a})
    \end{equation*}
    for every $a \in
    A$. So we have
    \begin{equation*}
        b^\prec (A) = \sum_{a \in A} f(U_{\preceq a}) - f(U_{\preceq
        a}) \leq \sum_{a \in A} f(A_{\preceq a}) - f(A_{\preceq a}),
    \end{equation*}
    which is a telescoping sum that simplifies to $f(A)$ (using $f(\emptyset)
    = 0$).

    If $A = U$, the estimation is not necessary and we have
    \begin{equation*}
        b^\prec (U) = \sum_{a \in U} f(U_{\preceq a}) - f(U_{\preceq a}) = f(U).
    \end{equation*}

    So we've shown $b^\prec \in B_f$.

    Take any $c \in \R^U$. We will show that there exists a total order $\prec$
    for which $b^\prec$ lies in the face defined by $c$.

    Define total order $U = \{u_1, \dots, u_n\}$ such that $c(u_1) \geq \dots
    \geq c(u_n)$. Denote $c_i := c(u_i)$ and $U_i = \{u_1, \dots, u_i\}$ for every $i \in \{1,
    \dots, n\}$ (and $U_0 = \emptyset$).

    Take any $x \in B_f$.
    We will show that $c^T b^\prec \geq c^T x$.
    Define $d_i := c_i - c_{i+1}$. By the definition of the ordering, we
    have $d_i \geq 0$ for all $i \in \{1, \dots, n\}$. With some reordering
    (we also used this at some point during the lectures) we have
    \begin{equation*}
        c^T x = \sum^n_{i=1} c_i x_i = \sum^n_{j=1} d_j \sum^j_{i=1} x_i.
    \end{equation*}
    Because $x \in B_f$, $x(U_i) \leq f(U_i)$ and $x(U) = f(U)$. Putting the
    into above equation we obtain
    \begin{equation*}
        \begin{split}
            c^T x &\leq \sum^n_{j=1} d_j f(U_j) \\
            &= c_n f(U) + \sum^{n-1}_{j=1} (c_j - c_{j+1}) f(U_j) \\
            &= \sum^n_{j=1} c_j (f(U_{j}) - f(U_{j-1})) \\
            &= c^T b^\prec.
        \end{split}
    \end{equation*}
    So for every face $F$ of polyhedron, there exists a total order $\prec$,
    such that $b^\prec \in F$. If $F$ happens to be a singleton, i.e. a vertex,
    then $F = \{b^\prec\}$.

    Now we have to show that for every total order $\prec$, vector $b^\prec$ is
    a vertex. Define $c_i = n - i + 1$. Take any $x \in B_f$ with $c^T x \geq
    c^T b^\prec$.
    For every $i$ we have $x(U_i) \leq f(U_i)$.
    We have
    \begin{equation*}
        \begin{split}
            c^T x &= \sum^n_{j=1} \sum^j_{i=1} x_i \\
            &\leq \sum^n_{j=1} f(U_j) \\
            &\leq \sum^n_{i=1} c_i (f(U_i) - f(U_{i-1})) \\
        \end{split}
    \end{equation*}
    By our choice of $x$, we therefore have an equality at all steps. 
    That means inequalities $x(U_i) \leq f(U_i)$ must in fact be equalities for
    every $i$. From that we can explicitly deduce that $x = b^\prec$. So
    $b^\prec$ is a vertex.
\end{exercise}

\end{document}
